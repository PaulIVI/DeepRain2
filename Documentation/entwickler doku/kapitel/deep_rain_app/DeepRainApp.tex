\section{Getting Started}
Im ersten Schritt den Master aus dem Git auschecken (DeepRain2). So bekommt Ihr den finalen Stand des Projektes. 

\section{deep\_rain\_app Ordnerstruktur}
In diesem Ordner befindet sich der gesamte Code, der für jegliche Funktionalitäten der App benötigt wird. 
Dazu gehört der Eigentliche App Code (deep\_rain), die Cloudfunction, welche für die Pushbenachrichtigung benötigt wird (FlutterCloudFunction),
sowie der Serversimulator (deeprainFireBase). 

\subsection{deep\_rain}
Die relevanten Ordner sind lib und assets. 
\subsubsection{assets}
Hier befinden sich alle in der App benötigten Bilder, sowie andere statische Daten. 
Dazu gehören zum Beispiel die Listen mit hilfe welcher der Pixel in den Vorhersage PNGs berechnet werden kann (data). 

\subsubsection{lib}
In diesem Ordner befindet sich der komplette, selbst geschriebene, Flutter Code. 
\subsubsection*{DataObjects}
In diesem Ordner befinden sich die Klasse, welche die Vorhersage PNGs speichert (DataHolder), sowie
die Klasse die das Objekt beschreibt, welches in der Liste angezeigt wird (ForecastListItem).
\subsubsection*{global}
In diesem Ordner befinden sich zwei Datenklassen welche von der gesamten App verwendet werden. 
In der Datei GlobalValues befinden sich alle global gespeicherten Variablen.
In der Datei UIText befinden sich alle angezeigten Texte in verschiedenen Sprachen. 
\subsubsection*{screens}
Es gibt für jeden Screen eine Datei in dem Ordner screens. Diese werden über die Datei main.dart verwaltet. 
\subsubsection*{Services}
Database.dart kümmert sich um die Kommunktion mit der Datenbank und ist die einzige Klasse, welche direkt mit der Datenbank 
kommuniziert. Alle Datenbank aktionen laufen über diese Klasse. 
FindPixel.dart stellt eine Funktion zur Verfügung, mit welcher es möglich ist, den eigenen Pixel im Vorhersage PNG zu finden. 
ProvideForecastData stellt die aktuellen ForecastListItems für die Forecast List zur verfügung. 
In PushNotification.dart wird festgelegt, wie sich die App bei dem öffnen und erhalten einer Pushbenachrichtigung verhalten soll. 
\subsubsection*{Widgets}
Die beiden .dart Dateien in diesem Ordner legen fest, wie die ForecastListe aussehen soll. forecast\_list\_widget stellt 
dabei die gesamte Liste dar, forecast\_tile nur eine Zeile aus der Liste. 

\subsection{deepRainFireBase}
Database.py simuliert den Server, wenn dieser nicht aktiv ist. 
Es werden Regenvorhersagen in die Datenbank geladen und Push Benachrichtigungen getriggert.
Viele Teile dieser Klasse sind später auch Teil der verwendeten Pipeline geworden. 
Diese Klasse ist nur ein Werkzeug, um die weiterentwicklung der App zu vereinfachen. 

\subsection{FlutterCloudFunction}
In dem Ordner functions, befindet sich in der Datei index.js die eigentliche CloudFunktion.
Die Funktionsweise dieser Funktion ist in der Hauptdokumentation ausgeführt. 
Um die änderungen an der CloudFunktion zu veröffentlichen und anzuwenden, muss diese zuerst über die Konsole 
deployed werden. 

\section{DeepRain App Getting Started}
Prof. Dr. Oliver Duerr kann euch Zugriff auf das Projekt in Firebase geben. 
Von dort aus habt Ihr auf die gesamte Firebase zugriff und könnt die Datenbanken, CloudFunktionen, Verschlüsselung ect. verwalten. 
\subsection{Die App zum laufen bekommen(deep\_rain)}
Um den Code auf einem Endgerät auszuführen, muss Flutter installiert sein. 
Siehe dazu folgender Link: https://flutter.dev/docs/get-started/install
Dann das Projekt in einer beliebigen IDE öffnen, Smartphone per USB verbinden und Code ausführen. 
Wir haben immer mit IntelliJ Ultimate Idea gearbeitet. 
\subsection{Server Simulator zum laufen bekommen (deepRainFireBase)}
Es müssen die benötigten Pakete installiert werden. Im Anschluss kann der Server ohne weiteres verwendet werden.