\section{Pipeline}
Nachdem die einzelnen Komponenten für sich funktionieren gilt es sie in einer Pipeline zu kombinieren und so ein funktionierendes Gesamtkonzept zu erhalten.
Der dadurch entstehende Workflow lässt sich in die Bereiche Datenbeschaffung, Regenvorhersage sowie Datenaufbereitung und Bereitstellung aufteilen.
Diese werden im Folgenden erläutert.

\subsection{Datenbeschaffung}
Zum Training sowie zur Validierung der Netze konnten bisher historische Daten verwendet werden.
Um eine Praxistaugliche Wettervorersage zu berechnen, muss diese allerdings in der Zukunft liegen.
Die Input Daten der Netze müssen daher möglichst aktuell sein.
Hiefür wurden, wie bereits bei den Historischen Daten, Daten des DWDs genutzt.
Dieser stellt alle fünf Minuten aktuelle Binärradardaten bereit.
Das Skript prüft alle fünf Sekunden ob neue Daten bereit stehen, falls das der Fall ist, werden diese heruntergeladen und anschließend wie in Abschnitt \ref{die datenaufbereitung} erläutert in Bilder mit einer Auflösung von 900x900 Pixeln konvertiert.
Anschließend wird ein Auschnitt des Bildes genommen welcher gegebenenfalls skaliert werden kann.
So kann der Ort für die Spätere Wettervorersage leicht angepasst werden.

\subsection{Regenvorhersage}
Die Vorhersage selbst hängt stark von den eingesetzten Netzen ab.
Derzeit werden drei gleiche Netze eingesetz welche auf 10, 20 und 30 Minuten vorhersagen trainert wurden.
Als Input für jedes Netz werden die 5 aktuellsten Bilder verwendet.
Deren Werte werden normiert und anschließend in einen Batch mit der Shape (1, 96, 96, 5) gebracht.

\subsection{Datenaufbereitung und Bereitstellung}
Die berechnete Vorhersage hat einen Wertebereich von 0-255 und eine Auflösung von 64x64 Pixel.
Um die Vorhersage in der App richtig darstellen zu können muss diese noch weiter bearbeitet werden.
Um vorhersagen flexibel an verschiedenen Orten oder in verschiedenen Auflösungen bereitstellen zu können ohne die App anzupassen, stellt diese für jede Vorhersage ein PNG mit 900x900 Pixeln dar.
Daher muss der Ausschnitt mit der Vorhersage bzw. mit den historischen Daten an der richtigen Position des PNGs eingefügt werden.
Die Pixel außerhalb dieses Auschnitts sind transparent.
Die unterschiedlichen Regenstärken werden mit drei verschiedenen Blautönen dargestellt.
Da die Regenintensität wie in Abschnitt \ref{die datenaufbereitung} erläutert nicht gleichverteilt ist, wurden die Schwellwerte zwei und zehn gewählt. 
Zudem wird jedem historischen Bild sowie jeder Vorhersage eine Zeitpunkt zugewiesen welcher später auf dem Slider in der App visualisiert wird.
Nun können die fünf historischen Regenbilder sowie die drei Vorhersagen in der Firebase-Datenbank ersetzt und so der App, also dem Nutzer bereitgestellt werden. 
Der Vorgang wiederholt sich und es wird alle fünf Sekunden geprüft ob auf dem Server des DWDs neue Daten vorhanden sind. 

