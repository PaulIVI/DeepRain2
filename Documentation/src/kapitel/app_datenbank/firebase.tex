\subsection{Firebase}\label{firebase}
Firebase ist eine Entwicklungsplattform für mobile Apps. 
Diese stellt verschiedene Services zur Verfügung, welche es ermöglichen, effizient Apps für IOS und Android zu entwickeln. 
Die Firebase ist ein zentraler Baustein der Komponente und wird in jeder Unterkomponente verwendet, weshalb hier einführend 
ein Überblick gegeben werden soll. 
Firebase stellt eine Datenbank und einen Cloud-Speicher zur Verfügung, welche genutzt werden, um die User zu verwalten und die 
Vorhersage PNGs auf den Geräten anzuzeigen.
Des Weiteren werden die In-App Messaging Dienste von Firebase verwendet, um die Regenwarnungen in Form von Push Benachrichtigungen zu senden. 
Die genaue Funktionsweise wird in \ref{sec:Pushbenachrichtigungen} beschrieben.   
Firebase ist bis zu einem gewissen Punkt der Verwendung komplett kostenlos. 
Wird dieser Punkt überschritten, sind die Kosten von der tatsächlichen Nutzung abhängig. 
In der kostenlosen Version enthalten sind 1 GB Cloud-Speicher, von welchem aktuell nur ein Bruchteil für die 20 PNGs 
verwendet wird. Des Weiteren können am Tag bis zu 20.000 Dokumente geschrieben und 125.000 Dokumente gelesen werden. 
Durch die komplett überarbeitete Datenbank und Softwarearchitektur wurden die Datenbanknutzung so weit verringert, 
dass diese Limitierungen bei Weitem nicht erreicht werden sollten.