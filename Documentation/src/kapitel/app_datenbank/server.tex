\subsection{Server}\label{server}
Mit der Serverkomponente ist der Teil des Servers gemeint, der für die Kommunikation mit der Firebase verantwortlich ist. Dazu gehört das bereitstellen der aktuellsten Regendaten in Zahlen und Bildformat sowie das triggern von Push Benachrichtigungen. 
Da zu beginn des Projektes noch keine Vorhersagen von den Netzen zur Verfügung stand, wurde ein Programm entwickelt, dass den Server simuliert und zufällig generierte Regendaten in der Firebase speichert. Somit konnte die App unabhängig und parallel zu den Netzen entwickelt werden. 
Für alle vorhandenen Regionen in denen Device Tokens gespeichert sind, es also Nutzer in der Region gibt, wird über den jeweiligen Breiten und Höhengrad der dazugehörige Pixel für die Region berechnet.
Wenn der Server ein Vorhersage PNG bekommt, bei dem die Regenstärke über einem bestimmten Grenzwert liegt, wird ein Dokument in der Kollektion RainWarningPushNotification gespeichert, welches von der Cloud Funktion verwendet wird, um eine Regenwarnung an die Geräte zu schicken. 