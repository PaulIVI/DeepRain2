\section{Einleitung}
In Quasi allen Zweigen der Wirtschaft spielt das Wetter eine Rolle welche das Handeln und den wirtschaftlichen Erfolg beeinflusst.
Angefangen beim Bäcker, dessen Personalplanung und Brötchenmenge vom Grillwetter abhängt, 
bis zur Landwirtschaft, in welcher der Anbau und Erntezeitpunkt auf Grundlage der Wettervorhersagen getroffen werden muss. 
Außerdem finden Wettervorhersagen selbstverständlich auch im privaten Sektor große Anwendung und werden von den meisten Menschen täglich verwendet. 
Während die Vorhersage von Temperaturen und Sonnenstunden mit Physikalischen Modellen inzwischen sehr zuverlässig funktionieren, 
sind Regenvorhersagen verhältnismäßig schwer zu machen. Nicht umsonst werden Regenvorhersagen in den meisten Fällen nur mit Wahrscheinlichkeiten angegeben. 
Das liegt vor allem an der immens großen Anzahl von Faktoren welche den Regen beeinflussen. 
Laut dem Deutschen Wetterdienst sind aktuell nur 80\% der Regenvorhersagen korrekt, bei den Temperaturen hingegen sind es über 90\% \cite{SpiegelWetter}. 
Der Rechenaufwand für Wettervorhersagen ist Enorm aber aufgrund der beschriebenen Relevanz werden die nötigen Ressourcen eingesetzt um ausreichend gute Wettervorhersagen zu ermöglichen.
In Deutschland gibt es mit dem Deutschen Wetterdienst eine Behörde die meteorologischen Dienstleistungen für die Allgemeinheit erbringt.
Diese werden im Folgenden auch in Anspruch genommen.      
\noindent Neuronale Netzte können dabei helfen in diesen komplexen und teilweise chaotisch wirkenden Vorgängen Muster zu erkennen und so Regenvorhersagen zu berechnen. 
