\chapter*{Extended Abstract}

\begin{center}
	\begingroup
	\renewcommand*{\arraystretch}{1}
	\rowcolors{2}{white}{white}
	{\makeatletter	
		\begin{tabular}{p{3.2cm}p{9.6cm}}
			Topic: & \thema \\
			& \\
			Team members: & \verfasserA, \verfasserB, \verfasserC \\
			& \\
			Advisor: & \hoschschule \newline \institut \newline \prueferA \\
			& \\
		\end{tabular}
		
		\makeatother}
	\endgroup
\end{center}

\bigskip

\noindent

The goal of the present work is to examine whether it is possible to calculate a rainfall forecast with limited resources and to make it available to users. 
For the calculation of the rain forecast neural networks were used. 
The required historical and current radar data were obtained from the German Weather Service and then analyzed and processed. 
Furthermore, an app was developed in which the rain forecasts are visualized.  
It also offers the possibility to notify the user in case of imminent rain. 
All the code and the full length documentation can be found on GitHub: \url{https://github.com/PaulIVI/DeepRain2}.



