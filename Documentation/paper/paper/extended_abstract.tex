\chapter*{Extended Abstract}

\begin{center}
	\begingroup
	\renewcommand*{\arraystretch}{1}
	\rowcolors{2}{white}{white}
	{\makeatletter	
		\begin{tabular}{p{3.2cm}p{9.6cm}}
			Topic: & \thema \\
			& \\
			Team members: & \verfasserA, \verfasserB, \verfasserC \\
			& \\
			Advisor: & \hoschschule \newline \institut \newline \prueferA \\
			& \\
		\end{tabular}
		
		\makeatother}
	\endgroup
\end{center}

\bigskip

In this Paper we try to predict precipitation for a range of 35 minutes in an area around Constance.
Therefore we are using machine learning techniques and train a UNet on radar data images. 
Here we present the result of precipitation prediction as well with regression as with classification. 
Both approaches provide good results. Source code and full length documentation in german can be found at GitHub: \url{https://github.com/thgnaedi/DeepRain}.


\printbibliography[title={References}, heading=subbibliography]

